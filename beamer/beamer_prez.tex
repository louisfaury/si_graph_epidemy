\documentclass[t]{beamer}

%% ----------------------------------------------------- PACKAGES ----------------------------------------------------- %%
\usepackage{coolPrez}
\usepackage{multimedia}
\bibliography{bibfile.bib}
%% ---------------------------------------------------- DOCUMENT ---------------------------------------------------- %%

\begin{document}

%% TITLE PAGE
\title[Susceptible-Infected Epidemic in Graphs]{Susceptible-Infected Epidemic in Graphs}
\author[Louis Faury\& Gallois - Montbrun Gr�goire]{Louis Faury\\ Gallois - Montbrun Gr�goire}
\institute[MICRO506 - Stochastic Methods]{MICRO506 - Stochastic Methods}
\newenvironment{subenv}{\only{\setbeamercolor{local structure}{fg=red}}}{}
\titlegraphic{\includegraphics[width=0.5\linewidth]{title_pic}}
\titlepage

%1
\begin{frame}[t]
	\vspace{-3ex}
	\frametitle{Plan}
  	\tableofcontents
\end{frame}

\section{Theoretical Approach}
{
	\subsection{Population Model}
	{
		\frame[t]
		{
			\vspace{15pt}
			$\blacksquare$ Hold the following assumptions for true : 
			\begin{itemize}
				\item Each individual in the population is either \emph{susceptible} or \emph{infected}
				\item Once infected, an individual can't recover and tries to infect other individuals 
				\item An individual can affect any other individual in the population (\textcolor{red}{uniform topology}) 
			\end{itemize}
			\vspace{30pt}
			$\blacksquare$ Graph population representation : 
				\begin{equation}
					G=(V,E)
				\end{equation} 
				with $\vert V \vert = n\in\mathbb{N}^*$\\
				Uniform topology $\Rightarrow$ \textbf{complete graph}. 
		}
	}
	\subsection{Propagation Model}
	{
		\frame[t]
		{
			\vspace{20pt}
			$\blacksquare$ Propagation :
			\begin{itemize}
				\item Each infected node tries to infect one of its neighbor at the times of a \textcolor{red}{Poisson process} of intensity $\lambda>0$. 
				\item The node's Poisson processes are independent. 
				\item An infected node picks one of its neighbor uniformly at random when trying to infect. 
			\end{itemize}
			
			\vspace{30pt}
			$\blacksquare$ Let $X_t$ the number of infected node at time $t>0$. \\
			\hspace{30pt} $\to$ $\{X_t\}$ is a \textbf{Markov Jump Process} on $\mathbb{R}^+$. 
		}
		
		\frame[t]
		{
			$\blacksquare$ Markov Jump Process  on $E$ : 
			\begin{itemize}
				\item If, $\forall x,y\in E$ : 
					\begin{equation}
						p_{xy}(t) = \mathbb{P}(X_{t+s}=y \, \vert \, X_s = x)
					\end{equation}
					then 
					\begin{equation}
						\lim_{t\to 0}p_{xx}(t) = 1
					\end{equation}
				\item The process remains at each stage for a strictly positive time with probability $1$. 
				\item We define the jump process's  \textbf{infinitesimal generator} as, $\forall x\neq y\in E$ 
				\begin{equation}
					q_{xy}= \lim_{h\to 0} \frac{p_{xy}(h)}{h}
				\end{equation}
				\item It defines the process \textcolor{red}{transitions rates}
				\item Ex : queuing systems !
			\end{itemize}
		}
		
		\frame[t]
		{
			$\blacksquare$ Back to our propagation model : 
			\begin{itemize}
				\item Since :
					\begin{equation}
						\begin{aligned}
							&p_{x,x-i}(t) = 0 \quad \forall{i>0}\\
							&p_{x,x+1}(h) \propto h \\
							&p_{x,x+1+i}(h) \propto h^i \quad \forall i >0
						\end{aligned}
					\end{equation}
				\item Only non-zero transition rate is $q_{x,x+1}$
				\item If $X_t = x>0$, the next time before an infection intent is a random variable defined as : 
					\begin{equation}
						\tau_x = \min_{i=1,\hdots,x} \eps_i \quad \text{ where } \eps_i \sim Exp(\lambda) \text{ (iid) }
					\end{equation}
					Therefore
					\begin{equation}
						\tau_x \sim Exp(\lambda x)
					\end{equation}
			\end{itemize}
		}
		
		\frame[t]
		{
		$\blacksquare$ Back to our propagation model : 
			\begin{itemize}
				\item The transition succeeds if the node picks a suspectible neighbor, which happens with probability $\frac{n-x}{n-1}$ (uniform) 
				\item Therefore : 
					\boxedeq{red}
					{
						q_{x,x+1} = \lambda x \frac{n-x}{n-1}
					}
				\item Let $X_0=1$. The time before $m$ individuals are infected is defined as : 
				\begin{equation}
					T_m = \sum_{x=1}^{m-1} \frac{n-1}{\lambda x (n-x)} \eps_i
				\end{equation}
				where 
				\begin{equation}
					\eps_i \overset{i.i.d}{\sim} Exp(1)
				\end{equation}	
			\end{itemize}
		}
	}
	\subsection{Time before complete infection}
	{
		\frame[t]
		{
			$\blacksquare$ Time before complete infection : $T_n$ which first moment is given by :
			\boxedeq{red}
			{
				\E{T_n} = \sum_{x=1}^n \frac{n-1}{\lambda x(n-x)} 
			}
			
			$\blacksquare$ Large population limit : 
			\begin{equation}
				\E{T_n} = \frac{2}{\lambda}(\log(n) + \gamma + o(1))
			\end{equation}
			scales with $\log(n)$ !\\
			
			$\blacksquare$ Fluctuation : if $S_n = \lambda(T_n - \E{T_n})$ : 
			\begin{equation}
				\mathbb{P}(S_n\geq t) \leq e^{-\theta t}C_\theta, \quad \forall\theta\in[0,\frac{1}{2}]
			\end{equation}
			$\rightarrow$ exponential control around the mean value !
			
		}
		
		\frame[t]
		{
			$\blacksquare$ Simulation 
			\begin{center}
				\movie[width=6cm,height=6cm,poster,autostart,borderwidth=1pt,loop]{}{complete_graph.avi}
			\end{center}
			\centering
			\vspace{-10pt}
			Hyper-parameters : $n=30$, $\lambda=1$. 
		}
		
		\frame[t]
		{
			$\blacksquare$ Simulation 
			\begin{figure}[h!]
				\begin{center}
					\includegraphics[width=0.5\linewidth]{complete_infection_time}
					\caption{Mean infection time as a function of $\vert V \vert$}
				\end{center}
			\end{figure}
			$\lambda =1$ and $\lambda_{MLE} = 0.94$ ! 
		}
		
		\frame[t]
		{
			$\blacksquare$ Simulation 
			\begin{figure}[h!]
				\begin{center}
					\includegraphics[width=0.5\linewidth]{fluctuation_db}
					\caption{Fluctuation distribution : $\mathbb{P}(S_n\geq t)$} 
				\end{center}
			\end{figure}
			
			\hspace{40pt} $\color{red}\boldsymbol{\rightarrow}$ Exponentially bounded tail
		}

	}
}

\section{Applications}
{
	\subsection{Epidemic Propagation}
	{
		\frame[t]
		{
			\vspace{20pt}
			$\blacksquare$ Epidemic Propagation : 
			\vspace{10pt}
			\begin{itemize}
			\setlength\itemsep{2em}
				\item Modelisation of simple epidemics in population 
				\item Simplistic model : 
					\begin{itemize}
						\item<sub@1> Nodes can't recover from being infected 
						\item<sub@1> A complete graph is not realistic 
					\end{itemize}
				\item What happens if we introduce recovering / dying nodes ? 
				\item How does the topology of the graph impacts the last results ?
			\end{itemize}
		}
	}
	\subsection{Sensor Networks}
	{
		\frame[t]
		{
			\vspace{30pt}
				$\blacksquare$ Sensor Network Modelisation (All to All propagation) : 
				\vspace{10pt}
				\begin{itemize}
					\setlength\itemsep{2em}
					\item Modelling a network of sensors / agents 
					\item Each agent tries to pass a bit of information to all others 
					\item Control over the time before every information has been propagated to every agent 
					\item Previous results still holds : scales as $\log{n}$
				\end{itemize}
		}
	}
}


\section{Model refinement}
{
	\frame[t]
	{
		\vspace{20pt}
		$\blacksquare$ Critics
		\vspace{10pt}
		\begin{itemize}
			\setlength\itemsep{1em}
			\item The SI approach is not realistic !
			\item More realistic models : SIS (Susceptible-Infected-Susceptible), SIR (Susceptible-Infected-Removed)
			\item Complete topology assumption does not hold (sparser graph, community)
		\end{itemize}
	}
	\subsection{SIS approach}
	{
		\frame[t]
		{
			$\blacksquare$ SIS (Susceptible-Infected-Susceptible approach) on general topology: 
			\begin{itemize}
				\item Let $G=(V,E)$ a graph with adjacency matrix $A \in \mathcal{M}_{\vert V \vert }(\mathbb{R})$
					\begin{equation}
						A = (\delta _{(v_i,v_j)\in E})_{i,j}
					\end{equation}
				\item $\{X(t)\}_t \in\{0,1\}^{\vert V \vert }$ our Markovian jump process. 
				\item Let : 
					\begin{equation}
						\begin{aligned}
							&\beta    &\text{ : infection rate} \\
							&\delta   &\text{ : remission rate} \\
							&e_i &=(\delta_{ji})_{j\in\{1,\hdots,V\}} 
						\end{aligned}
					\end{equation}
				\item Then the only non-zero infinitesimal generators are : 
					\begin{equation}
						\left\{
						\begin{aligned}
							&q_{x,x+e_i} = \beta \mathds{1}_{x_i=0} \sum_{j\in V}A_{i,j}x_j \\
							&q_{x,x-e_i} = \delta x_i 
						\end{aligned}\right.
					\end{equation}
			\end{itemize}
		}
		
		\frame[t]
		{
			$\blacksquare$ Simulation : 
			
			\begin{center}
				\movie[width=6cm,height=6cm,poster,autostart,borderwidth=1pt,loop]{}{sparse_graph.avi}
			\end{center}
			\centering
			\vspace{-10pt}
			Hyper-Parameters : $\delta$  = 0.5, $\beta = 0.25$, $\rho_G\simeq 4$
		}

				\frame[t]
		{
			$\blacksquare$ Simulation : 
			
			\begin{center}
				\movie[width=6cm,height=6cm,poster,autostart,borderwidth=1pt,loop]{}{sparse_graph_2.avi}
			\end{center}
			\centering
			\vspace{-10pt}
			Hyper-Parameters : $\delta$  = 0.5, $\beta = 0.5$, $\rho_G\simeq 4$
		}
		
		\frame[t]
		{
			$\blacksquare$ SIS (Susceptible-Infected-Susceptible approach) on general topology: 
			\vspace{5pt}
			\begin{itemize}
				\item $0^V$ is an \textcolor{red}{absorbant state} $\leftarrow$ time for absorption ? 
				\item Spectral radius of the graph : 
					\begin{equation}
						\rho_G = \max_{\lambda\in Sp(A)} \vert \lambda \vert 
					\end{equation}
				\item \textbf{Main absorption result} : If the graph is \emph{finite} and $t>0$
				\begin{equation}
					\mathbb{P}(T>t) \leq ne^{t(\beta \rho - \delta)}
				\end{equation}
				hence if $\color{red} \boldsymbol{\beta}\boldsymbol{\rho}_G \leq \boldsymbol{\delta}$
				\boxedeq{red}
				{
					\begin{aligned}
						\E{T} &= \int_{0}^{+\infty} \mathbb{P}(T>t)dt \\
							 &\leq \frac{\log{n}+1}{\delta - \beta \rho_G}
					\end{aligned}
				}
				
			\end{itemize}

		}
		
		\frame[t]
		{
			$\blacksquare$ Simulation 
			\begin{figure}[h!]
				\begin{center}
					\includegraphics[width=0.5\linewidth]{absorbing_time}
					\caption{Mean absorbing time as a function of $\vert V \vert$}
				\end{center}
			\end{figure}
			\centering
			$\delta =0.5$, $\beta = 0.025$, $\rho_G \leq 4$
		}
		
	}
	\subsection{Arbitrary graph topology}
	{
		\frame[t]
		{
			$\blacksquare$ SIS (Susceptible-Infected-Susceptible approach) on general topology: 
			\vspace{10pt}
			\begin{itemize}
				\setlength\itemsep{2em}
				\item More precise and complex results include the \textcolor{red}{\emph{graph's isoperimetric constant}} : $\forall m\in\{1,\hdots, n-1\} $
				\begin{equation}
					\eta_m(G)=  \inf_{S\subset V,\, \vert S\vert \leq m} \left\{\displaystyle \frac{\vert E(S,\bar{S})\vert }{\vert S \vert}\right\}
				\end{equation}
				\item Derive results of control for many different topologies (complete, hypercube, Erdos-Renyi's,$\hdots$)
				\item Model a \emph{volatile information} propagation on various network topologies 
				\item Paves the way for SIR model
			\end{itemize}
		}
	}
	
	\frame[t]
	{
		$\blacksquare$ \textcolor{red}{Summary}
		\vspace{20pt}  
		\begin{itemize}
			\setlength\itemsep{1em}
			\item Stochastic Process and Markov Jump Process on Graph 
			\item Propagation of a disease / information along communities 
			\item Control over mean propagation time and fluctuation around that mean 	
		\end{itemize}
		\pause
		\vspace{35pt}
		\centering 
		\Large{\textbf{ Thank you for your attention ! }}
		
	}
}

\end{document}